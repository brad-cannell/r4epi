\documentclass[]{book}
\usepackage{lmodern}
\usepackage{amssymb,amsmath}
\usepackage{ifxetex,ifluatex}
\usepackage{fixltx2e} % provides \textsubscript
\ifnum 0\ifxetex 1\fi\ifluatex 1\fi=0 % if pdftex
  \usepackage[T1]{fontenc}
  \usepackage[utf8]{inputenc}
\else % if luatex or xelatex
  \ifxetex
    \usepackage{mathspec}
  \else
    \usepackage{fontspec}
  \fi
  \defaultfontfeatures{Ligatures=TeX,Scale=MatchLowercase}
\fi
% use upquote if available, for straight quotes in verbatim environments
\IfFileExists{upquote.sty}{\usepackage{upquote}}{}
% use microtype if available
\IfFileExists{microtype.sty}{%
\usepackage{microtype}
\UseMicrotypeSet[protrusion]{basicmath} % disable protrusion for tt fonts
}{}
\usepackage{hyperref}
\hypersetup{unicode=true,
            pdftitle={R for Epidemiology},
            pdfauthor={Brad Cannell},
            pdfborder={0 0 0},
            breaklinks=true}
\urlstyle{same}  % don't use monospace font for urls
\usepackage{natbib}
\bibliographystyle{apalike}
\usepackage{longtable,booktabs}
\usepackage{graphicx,grffile}
\makeatletter
\def\maxwidth{\ifdim\Gin@nat@width>\linewidth\linewidth\else\Gin@nat@width\fi}
\def\maxheight{\ifdim\Gin@nat@height>\textheight\textheight\else\Gin@nat@height\fi}
\makeatother
% Scale images if necessary, so that they will not overflow the page
% margins by default, and it is still possible to overwrite the defaults
% using explicit options in \includegraphics[width, height, ...]{}
\setkeys{Gin}{width=\maxwidth,height=\maxheight,keepaspectratio}
\IfFileExists{parskip.sty}{%
\usepackage{parskip}
}{% else
\setlength{\parindent}{0pt}
\setlength{\parskip}{6pt plus 2pt minus 1pt}
}
\setlength{\emergencystretch}{3em}  % prevent overfull lines
\providecommand{\tightlist}{%
  \setlength{\itemsep}{0pt}\setlength{\parskip}{0pt}}
\setcounter{secnumdepth}{5}
% Redefines (sub)paragraphs to behave more like sections
\ifx\paragraph\undefined\else
\let\oldparagraph\paragraph
\renewcommand{\paragraph}[1]{\oldparagraph{#1}\mbox{}}
\fi
\ifx\subparagraph\undefined\else
\let\oldsubparagraph\subparagraph
\renewcommand{\subparagraph}[1]{\oldsubparagraph{#1}\mbox{}}
\fi

%%% Use protect on footnotes to avoid problems with footnotes in titles
\let\rmarkdownfootnote\footnote%
\def\footnote{\protect\rmarkdownfootnote}

%%% Change title format to be more compact
\usepackage{titling}

% Create subtitle command for use in maketitle
\providecommand{\subtitle}[1]{
  \posttitle{
    \begin{center}\large#1\end{center}
    }
}

\setlength{\droptitle}{-2em}

  \title{R for Epidemiology}
    \pretitle{\vspace{\droptitle}\centering\huge}
  \posttitle{\par}
    \author{Brad Cannell}
    \preauthor{\centering\large\emph}
  \postauthor{\par}
      \predate{\centering\large\emph}
  \postdate{\par}
    \date{2020-03-03}

\usepackage{booktabs}

\begin{document}
\maketitle

{
\setcounter{tocdepth}{1}
\tableofcontents
}
\hypertarget{welcome}{%
\chapter*{Welcome}\label{welcome}}
\addcontentsline{toc}{chapter}{Welcome}

Welcome to R for Epidemiology!

I'm going to start the book by writing down some basic goals that underlie the construction and content of this book. I'm writing this for you, the reader, but also to hold myself accountable as I write. So, feel free to read if you are interested or skip ahead if you aren't.

The goals of this book are:

\begin{enumerate}
\def\labelenumi{\arabic{enumi}.}
\item
  Make this writing as accessible and practically useful as possible without stripping out all of the complexity that makes doing epidemiology in real life a challenge. In other words, I'm going to try to give you all the tools you need to \emph{do} epidemiology in ``real world'' (as opposed to ideal) conditions without providing a whole bunch of extraneous (often theoretical) stuff that detracts from \emph{doing}. Having said that, I will strive to add links to the other (often theoretical) stuff for readers who are interested.
\item
  Teach you to accomplish tasks, rather than teach you to use functions. In many R texts, the focus in on learning all the things a function, or set of related functions, \emph{can} do. It's then up to you, the reader, to sift through all of these capabilities and decided which, if any, of the things that \emph{can} be done will accomplish the tasks that you are actually trying to accomplish. Instead, I will strive to start with the end in mind. What is the task we are actually trying to accomplish? What are some functions/methods I could use to accomplish that task? What are the strengths and limitations of each?
\item
  Where possible, we will start each concept with the end result and then deconstruct how we arrived at that result. I find that it is easier for me to understand new concepts when learning them as a component of a final product.
\item
  Where possible, we will learn concepts with data instead of (or alongside) mathematical formulas and text descriptions. I find that it is easier for me to understand new concepts by seeing them in action.
\end{enumerate}

\hypertarget{part-getting-started}{%
\part{Getting Started}\label{part-getting-started}}

\hypertarget{course-overview}{%
\chapter{Course overview}\label{course-overview}}

Blah, blah, blah. Here is what the course is about.

\hypertarget{installing-r-and-rstudio}{%
\chapter{Installing R and RStudio}\label{installing-r-and-rstudio}}

Here's how to do it.

\hypertarget{what-is-r}{%
\chapter{What is R?}\label{what-is-r}}

Here's what it is\ldots{}

\hypertarget{speaking-rs-language}{%
\chapter{Speaking R's language}\label{speaking-rs-language}}

Here's how to speak it\ldots{}

\hypertarget{navigating-the-rstudio-interface}{%
\chapter{Navigating the RStudio interface}\label{navigating-the-rstudio-interface}}

Here's how to navigate it\ldots{}

\hypertarget{lets-get-programming}{%
\chapter{Let's get programming}\label{lets-get-programming}}

Let's do it!

\hypertarget{asking-questions}{%
\chapter{Asking questions}\label{asking-questions}}

Use a repex, please!

\hypertarget{part-descriptive-analysis}{%
\part{Descriptive Analysis}\label{part-descriptive-analysis}}

\hypertarget{descriptive-analysis}{%
\chapter{Descriptive analysis}\label{descriptive-analysis}}

Bring over PowerPoint and all the Rmd files from last year's course.

\hypertarget{part-appendix}{%
\part{Appendix}\label{part-appendix}}

\hypertarget{appendix-a-vocabulary-words}{%
\chapter*{Appendix A: Vocabulary words}\label{appendix-a-vocabulary-words}}
\addcontentsline{toc}{chapter}{Appendix A: Vocabulary words}

\begin{itemize}
\tightlist
\item
  Model\\
\item
  Distribution\\
\item
  Sample\\
\item
  Study design\\
\item
  Primary and secondary data\\
\item
  Observe-sort of implies that we're counting
\end{itemize}

\bibliography{book.bib,packages.bib}


\end{document}
